%%This is a very basic article template.
%%There is just one section and two subsections.
\documentclass{article}
\title{Integracja aplikacji oparta o przesy�anie wiadomo�ci}
\author{Konrad Starzyk \\kstarzyk@stud.elka.pw.edu.pl}

\usepackage{graphicx}
\usepackage{polski}
\usepackage[cp1250]{inputenc}

\begin{document}
\maketitle

\section{Wst�p}

Integracja jest problemem na kt�ry natrafia ka�dy programista gdy tworzona
przez niego aplikacja ma wsp�pracowa� z innymi, dzia�aj�cymi ju� systemami.
Cz�sto z rozmaitych wzgl�d�w jest to problem niebanalny, czy to ze wzgl�du na
r�ne technologie w kt�rych dwa systemy zosta�y stworzone, czy te� ch��
zachowania integralno�ci ka�dego z nich i niedopuszczenia do niskopoziomowych
modyfikacji.
Problem integracji istnieje tak d�ugo, jak istniej� niekompatybilne rozwi�zania
- aplikacje, kt�re nie by�y przewidziane do wsp�pracy mi�dzy sob�, a pojawia
si� potrzeba ich wsp�pracy.
Z czasem pojawi�o kilka powszechnie stosowanych sposob�w wspomnianego problemu,
z kt�rych ka�dy ma swoje zastosowania.
\begin{itemize} 
    \item Transfer plik�w
	\item Wsp�lna baza danych
	\item RPC
	\item Wiadomo�ci
\end{itemize}


\subsection{Another subtitle}

More plain text.









\begin{thebibliography}{99}
\bibitem{Entpatterns} Gregor Hophe,Bobby Woolf: \emph{Enterprise Integration
Patterns}, The Addison-Wesley Signaure Series 2003
\bibitem{techfaq360}: \emph{Spring Tutorial},
http://www.techfaq360.com/tutorial/spring/
\end{thebibliography}

\end{document}
