\pdfoutput=1
\pdfcompresslevel=9
\pdfinfo  
{
    /Author (Wojciech Klicki,Konrad Starzyk)
    /Title ()
    /Subject (Integracja system�w mobilnych) 
    /Keywords (Integracja mobilna, Systemy Enterprise) 
} 
%\documentclass[a4paper,polish,onecolumn,oneside,floatssmall,11pt,titleauthor,wide,openright]{mwrep}
%\usepackage[scale={0.7,0.8},paper=a4paper,twoside]{geometry}

\documentclass[a4paper,polish,onecolumn,oneside,12pt,wide,floatssmall]{mwrep}
\usepackage{polish}
\usepackage{amsmath}  
\usepackage{amsfonts}  
%\usepackage{amssymb}
\usepackage{amsthm}    
\usepackage{bookman} 
\usepackage{fancyhdr}    
\usepackage[cp1250]{inputenc} 
\usepackage{makeidx}  
\makeindex

\usepackage{geometry}
\usepackage{t1enc}
% \usepackage[pdftex, bookmarks]{hyperref}
\usepackage[pdftex, bookmarks=false, hyperref=false]{hyperref}
\def\url#1{{ \tt #1}}

\usepackage{listings}

% marginesy
\textwidth\paperwidth
\advance\textwidth -60mm %reguluje prawy margines
\oddsidemargin-0.9in
\advance\oddsidemargin 38mm
\evensidemargin-0.9in
\advance\evensidemargin 38mm
\topmargin -1in
\advance\topmargin 25mm
\setlength\textheight{45\baselineskip} %ilosc linii - pozwala na regulacje
									   %dolngo marginesu
\addtolength\textheight{\topskip}
\marginparwidth15mm

\clubpenalty=10000 % to kara za sierotki
\widowpenalty=10000 % nie pozostawia wd�w
\brokenpenalty=10000 % nie dzieli wyraz�w pomi�dzy stronami 
\sloppy

\tolerance4500
\pretolerance250
\hfuzz=1.5pt
\hbadness1450

% �YWA PAGINA
\renewcommand{\chaptermark}[1]{\markboth{\scshape\small\bfseries \
#1}{\small\bfseries \ #1}}
\renewcommand{\sectionmark}[1]{\markboth{\scshape\small\bfseries\thesection.\
#1}{\small\bfseries\thesection.\ #1}}
\renewcommand{\headrulewidth}{0.5pt}
\renewcommand{\footrulewidth}{0.pt}
\pagestyle{uheadings}

\usepackage[pdftex]{color,graphicx}

%\usepackage[utf8]{inputenc}
%\usepackage[polish]{babel}

% \textheight232mm
%%% \setlength{\textwidth}{\textwidth}
% \setlength{\oddsidemargin}{\evensidemargin}
% \setlength{\evensidemargin}{0.3cm}
\usepackage[sort, compress]{cite}

%\usepackage{multibib}
%\newcites{bk,st,doc,web}{Ksi��ki i~artyku�y,Standardy i~zalecenia,Dokumentacja produkt�w,Publikacje i~serwisy internetowe}

\theoremstyle{definition}
\newtheorem{defn}{Definicja}[section]
\newtheorem{conj}{Teza}[section]
\newtheorem{conjmain}{Teza}
\newtheorem{exmp}{Przyk�ad}[section]

\theoremstyle{plain}% default
\newtheorem{thm}{Twierdzenie}[section]
\newtheorem{lem}[thm]{Lemat}
\newtheorem{prop}[thm]{Hipoteza}
\newtheorem*{cor}{Wniosek}

\theoremstyle{remark}
\newtheorem*{rem}{Uwaga}
\newtheorem*{note}{Uwaga}
\newtheorem{case}{Przypadek}

\definecolor{ListingBackground}{rgb}{0.95,0.95,0.95}

\begin{document}

% kody �r�d�owe wplatane w tekst
\lstdefinestyle{incode}
{
basicstyle={\footnotesize},
keywordstyle={\bf\footnotesize\color{blue}},
commentstyle={\em\footnotesize\color{magenta}},
numbers=left, 
stepnumber=5, 
firstnumber=1,
numberfirstline=true,
numberblanklines=true,
numberstyle={\sf\tiny}, 
numbersep=10pt, 
tabsize=2,
xleftmargin=17pt,
framexleftmargin=3pt,
framexbottommargin=2pt,
framextopmargin=2pt,
framexrightmargin=0pt,
showstringspaces=true,
backgroundcolor={\color{ListingBackground}},
extendedchars=true,
% title=\lstname,
captionpos=b,
% abovecaptionskip=1pt,
% belowcaptionskip=1pt,
frame=tb,
framerule=0pt, 
}
 
% kody �r�d�owe z podpisem
\lstdefinestyle{outcode}
{
basicstyle={\footnotesize},
keywordstyle={\bf\footnotesize\color{blue}},
commentstyle={\em\footnotesize\color{magenta}},
numbers=left, 
stepnumber=5, 
firstnumber=1,
numberfirstline=true,
numberblanklines=true,
numberstyle={\sf\tiny}, 
numbersep=10pt, 
tabsize=2,
xleftmargin=17pt,
framexleftmargin=3pt,
framexbottommargin=2pt,
framextopmargin=2pt,
framexrightmargin=0pt,
showstringspaces=true,
backgroundcolor={\color{ListingBackground}},
extendedchars=true,
% title=\lstname,
captionpos=b,
% abovecaptionskip=1pt,
% belowcaptionskip=1pt,
frame=tb,
framerule=0.1pt, 
}

\renewcommand*\lstlistingname{Wydruk}
\renewcommand*\lstlistlistingname{Spis wydruk�w}

\pagenumbering{roman}
\renewcommand{\baselinestretch}{1.0}
\raggedbottom

\begin{titlepage}
    % Strona tytu�owa
    \vbox to\textheight{\hyphenpenalty=10000
    \begin{center}
	\begin{tabular}{p{107mm} p{9cm}}
	    \begin{minipage}{9cm}
	      \begin{center}
	      Politechnika Warszawska \\
	      Wydzia�� Elektroniki i~Technik Informacyjnych \\
	      Instytut Informatyki
	      \end{center}
	    \end{minipage}
	    &
	    \begin{minipage}{8cm}
	    \begin{flushleft}
	     \footnotesize
	      Rok akademicki 2008/2009
	    \vspace*{2.75\baselineskip}
	    \end{flushleft}
	    \end{minipage} \\
	\end{tabular}
	\vspace*{3.75\baselineskip}
	\par\vspace{\smallskipamount}
	\vspace*{2\baselineskip}{\LARGE Praca dyplomowa magisterska\par}
	\vspace{3\baselineskip}{\LARGE\strut Wojciech Klicki\\Konrad Starzyk\par}
	\vspace*{2\baselineskip}{\huge\bfseries Integracja technologii mobilnych i system�w klasy Enterprise\par}
	
	\vspace*{7\baselineskip}
	\hfill\mbox{}\par\vspace*{\baselineskip}\noindent
	\begin{tabular}[b]{@{}p{3cm}@{\ }l@{}}
	    {\large\hfill } & {\large }
	\end{tabular}
	\hfill
	\begin{tabular}[b]{@{}l@{}}
	Opiekun pracy: \\[\smallskipamount]
	{\large mgr in�. Piotr Salata}
	\end{tabular}\par
	\vspace*{4\baselineskip}
    \begin{tabular}{p{\textwidth}}
    \begin{flushleft}
	\begin{minipage}{7cm}    
	Ocena \dotfill 
	\par\vspace{1.6\baselineskip}
	\dotfill 
	\par\noindent
	\centerline{\footnotesize Podpis Przewodnicz�cego} \par
	\centerline{\footnotesize Komisji Egzaminu Dyplomowego}\par
	\end{minipage}
    \end{flushleft}
    \end{tabular}
    \end{center}}

    % �yciorys Wojtek
    \newpage\thispagestyle{empty}
    \begin{tabular}{p{5cm} p{12cm}}
    \begin{minipage}{5cm}
    \center
    \includegraphics[height=6.5cm,width=4.5cm]{img/dyplom2.jpg} 
    \end{minipage}
    &
    \begin{minipage}{12cm}
    \begin{flushleft}
    \par\noindent\vspace{1\baselineskip} 
    \begin{tabular}[h]{l l}
    {\normalsize\it Specjalno��:} & Informatyka -- \\
    & In�ynieria oprogramowania \\
    & i~systemy informacyjne 
    \end{tabular}
    \par\noindent\vspace{1\baselineskip} 
    \begin{tabular}[h]{l l}
    {\normalsize\it Data urodzenia:} & {\normalsize 1 stycznia 1980~r.} 
    \end{tabular}
    \par\noindent\vspace{1\baselineskip}
    \begin{tabular}[h]{l l}
    {\normalsize\it Data rozpocz�cia studi�w:} & {\normalsize 1 pa�dziernika
    2004 r.}
    \end{tabular}
    \par\noindent\vspace{1\baselineskip}
    \end{flushleft}
    \end{minipage}
    \end{tabular}
    \vspace*{1\baselineskip}
    \begin{center}
	{\large\bfseries �yciorys}\par\bigskip
    \end{center}
    
    \indent
    Nazywam si�  \ldots.
    \par
    \vspace{2\baselineskip}
    \hfill\parbox{15em}{{\small\dotfill}\\[-.3ex]
    \centerline{\footnotesize podpis studenta}}\par
    \vspace{3\baselineskip}
    \begin{center}
 	{\large\bfseries Egzamin dyplomowy} \par\bigskip\bigskip
    \end{center}
    \par\noindent\vspace{1.5\baselineskip}
    Z�o�y� egzamin dyplomowy w dn. \dotfill 
    \par\noindent\vspace{1.5\baselineskip}
    Z wynikiem \dotfill 
    \par\noindent\vspace{1.5\baselineskip}
    Og�lny wynik studi�w \dotfill
    \par\noindent\vspace{1.5\baselineskip}
    Dodatkowe wnioski i uwagi Komisji \dotfill
    \par\noindent\vspace{1.5\baselineskip}
    \dotfill
    
    
     % �yciorys Wojtek
    \newpage\thispagestyle{empty}
    \begin{tabular}{p{5cm} p{12cm}}
    \begin{minipage}{5cm}
    \center
    \includegraphics[height=6.5cm,width=4.5cm]{img/dyplom1.jpg} 
    \end{minipage}
    &
    \begin{minipage}{12cm}
    \begin{flushleft}
    \par\noindent\vspace{1\baselineskip} 
    \begin{tabular}[h]{l l}
    {\normalsize\it Specjalno��:} & Informatyka -- \\
    & In�ynieria oprogramowania \\
    & i~systemy informacyjne 
    \end{tabular}
    \par\noindent\vspace{1\baselineskip} 
    \begin{tabular}[h]{l l}
    {\normalsize\it Data urodzenia:} & {\normalsize 1 stycznia 1980~r.} 
    \end{tabular}
    \par\noindent\vspace{1\baselineskip}
    \begin{tabular}[h]{l l}
    {\normalsize\it Data rozpocz�cia studi�w:} & {\normalsize 1 pa�dziernika
    2004 r.}
    \end{tabular}
    \par\noindent\vspace{1\baselineskip}
    \end{flushleft}
    \end{minipage}
    \end{tabular}
    \vspace*{1\baselineskip}
    \begin{center}
	{\large\bfseries �yciorys}\par\bigskip
    \end{center}
    
    \indent
    Nazywam si�  \ldots.
    \par
    \vspace{2\baselineskip}
    \hfill\parbox{15em}{{\small\dotfill}\\[-.3ex]
    \centerline{\footnotesize podpis studenta}}\par
    \vspace{3\baselineskip}
    \begin{center}
 	{\large\bfseries Egzamin dyplomowy} \par\bigskip\bigskip
    \end{center}
    \par\noindent\vspace{1.5\baselineskip}
    Z�o�y� egzamin dyplomowy w dn. \dotfill 
    \par\noindent\vspace{1.5\baselineskip}
    Z wynikiem \dotfill 
    \par\noindent\vspace{1.5\baselineskip}
    Og�lny wynik studi�w \dotfill
    \par\noindent\vspace{1.5\baselineskip}
    Dodatkowe wnioski i uwagi Komisji \dotfill
    \par\noindent\vspace{1.5\baselineskip}
    \dotfill

    % Streszczenie
    \newpage\thispagestyle{empty}
    \vspace*{2\baselineskip}
    \begin{center}
	{\large\bfseries Streszczenie}\par\bigskip
    \end{center}
    
    {\itshape 
    Praca ta prezentuje \ldots}
    \vspace*{1\baselineskip}
    
    \noindent{\bf S�owa kluczowe}: {\itshape s�owa kluczowe.}
    \par
    \vspace{4\baselineskip}
    \begin{center}
	{\large\bfseries Abstract}\par\bigskip
    \end{center}
    \noindent{\bf Title}: {\itshape Thesis title.}\par
    \vspace*{1\baselineskip}
    {\itshape 
    This thesis describes \ldots}
    \vspace*{1\baselineskip}
       
    \noindent{\bf Key words}: {\itshape key words.}
   
\end{titlepage}

% ex: set tabstop=4 shiftwidth=4 softtabstop=4 noexpandtab fileformat=unix filetype=tex encoding=utf-8 fileencodings= fenc= spelllang=pl,en spell:



\linespread{1.5}%
\selectfont

\tableofcontents
% \addcontentsline{toc}{chapter}{{Przedmowa1}{vii}}{vii}

% \chapter*{Spis tablic, rysunk�w i~wydruk�w}
% \listoftables
% \listoffigures
% \lstlistoflistings 

%\setlength{\baselineskip}{7mm}
\newpage
\pagenumbering{arabic}
\setcounter{page}{1}


\chapter{Integracja w �rodowiskach Enterprise}
Oprogramowanie Enterprise jest poj�ciem do�� szerokim, opisuj�cym systemy
przeznaczone dla przedsi�biorstw, kt�rych dzia�anie odwzorowuje
zachodz�ce procesy biznesowe. Niekiedy poj�cie to odnosi si� do oprogramowania
pisanego na zam�wienie lub te� do rozbudowanych pakiet�w wspieraj�cych
okre�lone czynno�ci, np. kontakty z klientami lub ksi�gowo��. Rzadko si� jednak
zdarza, by istnia� jeden system kt�ry potrafi�by spe�ni� wymagania
klienta, a nawet je�eli, z r�nych przyczyn firma mo�e nie chcie� go
wdro�y�. Tak wi�c nawet w jednym przedsi�biorstwie cz�sto
mo�na si� spotka� z sytuacj� w kt�rej dzia�a wiele niezale�nych aplikacji, 
nierzadko pisanych przez r�ne firmy, kt�re musz� si� ze sob� komunikowa� w
celu wymiany danych.  
\newline 
Jeszcze do nie dawna m�wi�c o integracji mieli�my na my�li w�a�ciwie
wy��cznie serwery aplikacyjne. Obecnie obowi�zuj�cym trendem jest post�puj�ca
miniaturyzacja i wzrost mocy obliczeniowej urz�dze� mobilnych, kt�re
udost�pniaj� klientom us�ugi dost�pne do tej pory wy��cznie za po�rednictwem komputera stacjonarnego.
Tworz�c aplikacj� na urz�dzenia przeno�ne, kt�ra b�dzie wsp�pracowa�a z
istniej�cymi ju� systemami, napotykamy jednak na nowe problemy, wynikaj�ce ze
specyfiki �rodowiska mobilnego.

\section{Potrzeba integracji}

Potrzeba integracji pojawia si�, gdy pojawia si� r�norodno��, kt�ra
wymaga dodatkowego nak�adu pracy sprowadzaj�cego j� do wsp�lnego poziomu.
Warto zauwa�y�, �e istniej� r�ne poziomy r�norodno�ci, a
rozwi�zania integracyjne mog� albo usuwa� t� r�norodno�� na poziomie
technicznym albo odpowiada� na potrzeby wynikaj�ce z r�norodno�ci na wy�szym
poziomie, na przyk�ad organizacyjnym. Mo�emy wi�c wyr�ni� nast�puj�ce poziomy
r�norodno�ci:
\begin{itemize}
  \item Infrastrukturalna
  \subitem - u�ycie r�nych narz�dzi i architektur
  \subitem - u�ycie r�nych standard�w i proces�w
  \item Dyscyplinarna
  \subitem  - oddzielne dzia�y wewn�trz firmy lub r�ne firmy
  \item Geograficzna
  \subitem - w ramach jednego kraju
  \subitem - w ramach wielu kraj�w/stref czasowych
  \item Zawarto�ci
  \subitem - rozdzielenie dzia��w operacyjnych, np. obs�uga klienta, logistyka,
  ksi�gowo��
  \subitem - podzia� pracy pomi�dzy dzia�y, np procesu kt�ry jest obs�ugiwany
  przez ka�dy z nich
\end{itemize}

W niniejszej pracy nie b�dziemy rozwa�a� przyczyn r�norodno�ci, przyjmujemy j�
raczej jako problem, kt�ry nale�y rozwi�za�. Jednak samo zwr�cenie uwagi
na list� mo�liwych przyczyn, sprawia, �e mo�emy sobie wyobrazi�, i� systemy
jakie spotykamy w przedsi�biorstwach du�ej skali cz�sto sk�adaj� si� z setek, je�li nie tysi�cy aplikacji wykonanych na zam�wienie przez zewn�trzne firmy lub przez wewn�trzne dzia�y informatyczne. Cz�� z tych aplikacji jest
tak zwanymi 'legacy systems' napisanymi w zapomnianych ju� j�zykach,
przygotowanymi pod platformy, kt�re cz�sto nie mog� liczy� ju� na wsparcie
producent�w. Wielokrotnie mo�na natrafi� na kombinacje tego typu aplikacji
dzia�aj�ce w r�nych warstwach w obr�bie r�nych system�w operacyjnych.

\subsection{Pe�nowarto�ciowe �rodowiska informatyczne}

Przyczyn le��cych u podstaw tak du�ego skomplikowania sytuacji w �rodowiskach
informatycznych typu Enterprise nale�y doszukiwa� si� w skomplikowaniu
zagadnienia jakim jest projektowanie du�ych aplikacji. Zbudowaniem jednej, posiadaj�cej
wszystkie niezb�dne funkcje aplikacji, kt�ra by�aby w stanie w pe�ni zaspokoi�
potrzeby du�ego przedsi�biorstwa. Firmy dzia�aj�ce na rynku ERP (Enterprise
Resource Planning) od lat nie ustaj� w wysi�kach by zbudowa� tego typu
aplikacj�. Jednak nawet najwi�ksze z nich wytworzy�y produkty, kt�re pokrywaj�
tylko cz�� niezb�dnych funkcji. Mo�na to zaobserwowa� na przyk�adzie system�w,
kt�re stanowi� punkty integracyjne we wsp�czesnych rozwi�zaniach.
Kolejnym powodem jest to, �e rozwi�zania, kt�re sk�adaj� si� z wielu ma�ych
zintegrowanych podsystem�w pozwalaj� na pewn� elastyczno�� w doborze
najlepszych rozwi�za� w zakresie danej dziedziny. Mog� wybra� takie
rozwi�zanie, kt�re jest w danej chwili najlepsze i stosunkowo najta�sze na
rynku. Nie s� zmuszeni do inwestowania w ca�o�ciowe rozwi�zania dostarczane
przez tylko jednego dostawc�. Co wi�cej opieranie si� na wielu podsystemach
wykonywanych przez r�ne firmy mo�e pozwoli� na pewne zr�wnoleglenie prac nad
wdro�eniem odr�bnych cz�ci systemu. W przypadku jednego du�ego systemu
potencjalnie prace nad nim mog�yby by� blokowane przez braki w zasobach
przedsi�biorstwa realizuj�cego zlecenie wdro�enia.\newline
Niestety takie podej�cie przynosi najlepsze efekty przy �cis�ym podziale
funkcjonalno�ci pomi�dzy produktami r�nych producent�w. W rzeczywisto�ci
pokusa budowania system�w, kt�re zabior� elementarn� funkcjonalno�� innej
cz�ci z zupe�nie innego pola, jest zbyt du�a i prowadzi do powstawania
rozwi�za�, kt�re nie maj� wyra�nego podzia�u na dziedziny.\newline
Przy rozwi�zaniach tego typu nale�y zwr�ci� uwag�, �e z punktu widzenia
u�ytkownika system enterprise, pomimo �e w rzeczywisto�ci mo�e sk�ada� si� z
wielu ma�ych, zintegrowanych ze sob� system�w, stanowi jedn� ca�o��. Na
przyk�ad u�ytkownik mo�e wys�a� zapytanie o stan konta oraz adres zamieszkania
klienta. W sytuacji gdy mamy do czynienia z systemem korporacyjnym takie
zapytanie mo�e wymaga� odwo�ania si� do dw�ch zupe�nie r�nych podsystem�w
(bilingowego oraz kontaktu z klientem). System cz�sto musi dokona� autentykacji
oraz autoryzacji u�ytkownika, sprawdzi� obci��enie serwer�w w celu wybrania
optymalnego do realizacji zlecenia. Tego typu proces mo�e w bardzo prosty
spos�b zaanga�owa� kilka system�w. Z punktu widzenia u�ytkownika jest to
pojedyncza transakcja.\newline
W celu zapewnienia poprawno�ci dzia�ania system�w, umo�liwienia wydajnej
wymiany danych oraz zapewnienia przezroczystego funkcjonowania proces�w
biznesowych, opisane powy�ej aplikacje musz� zosta� zintegrowane. Wraz z
integracj� pojawiaj� si� inne potrzeby takie jak zapewnienie bezpiecznej
wymiany danych czy te� umo�liwienie dost�pu do systemu z r�nych platform
zewn�trznych.

\subsection{Systemy mobilne}

Jedn� z podstawowych cech jakie musi posiada� system enterprise jest �atwo��
dost�pu do przechowywanym w nim informacji. Typowy u�ytkownik takiego systemu
chce mie� mo�liwo�� po��czenia si� z nim i pobrania konkretnych informacji o
ka�dej porze dnia czy nocy. Wraz z pojawieniem si� nowej generacji urz�dze�
mobilnych mo�liwym sta�o si� zrealizowanie tej potrzeby. Nowoczesne platformy
oferuj� niespotykan� wcze�niej moc oraz zasi�g, kt�re pozwalaj� na sta�y dost�p
do sieci firmowej z dowolnego niemal miejsca na ziemi. Us�ugi synchronizuj�ce
poczt� korporacyjn�, oferuj�c� pojedyncz� skrzynk� poczty przychodz�cej, s�
obecnie niezb�dnym minimum w ka�dej du�ej firmie. Nast�pnym etapem w rozwoju
jest umo�liwianie szybkiego i interaktywnego dost�pu do dzia�aj�cych w obr�bie
intranetu us�ug wewn�trznych. Ten nowy rodzaj potrzeby integracji - integracja
mobilna, stanowi zupe�nie nowy rodzaj wyzwania dla firm oferuj�cych narz�dzia
dla biznesu. Niesie ona ze sob� zupe�nie nowe zagadnienia oraz problemy, kt�re
nie by�y spotykane przy klasycznej integracji. Wymusza to wypracowanie zupe�nie
nowej metodologii przemys�owego wytwarzania oprogramowania zapewniaj�cego
zaspokojenie tej potrzeby.
  
\section{Klasyczne wyzwania integracji}

Integracja aplikacji w przedsi�biorstwach cz�sto nie jest prostym zadaniem.
Mimo, �e w niemal ka�dej du�ej firmie zachodzi potrzeba integracji i powsta�o
ju� wiele opracowa� na ten temat, nikt nie przedstawi� kompletnej listy
problem�w, kt�re nale�a�oby rozwa�y� podczas integrowania system�w. Dodatkowo,
wyzwania, kt�re stawia integracja, wykraczaj� nie tylko poza techniczne
rozwa�ania, ale tak�e rozci�gaj� si� na procedury biznesowe. Tak jak ju�
wspominali�my, przedstawienie kompletnej listy takich problem�w nie jest
mo�liwe ze wzgl�du na konieczno�� dokonania analizy ka�dego z przypadk�w. Mimo to spr�bujemy
przedstawi� te najbardziej typowe, zauwa�one i opisane, kt�re
koniecznie trzeba wzi�� pod uwag�.

\subsection{Wyzwania techniczne}

\paragraph{Fizyczna i logiczna odr�bno�� system�w,}
z kt�rej tak naprawd� wynika potrzeba integracji. Wprowadza to jednak konieczne
do rozwa�enia kwestie, takie jak:
\begin{itemize}
  \item wyb�r sposobu komunikacji mi�dzy systemami (o kt�rych wi�cej w
  rozdziale 1.4)
  \item wp�yw tego wyboru na bezpiecze�stwo i zgodno�� takiego rozwi�zania z
  polityk� bezpiecze�stwa firmy
\end{itemize}
  
\paragraph{Ograniczony dost�p do integrowanych aplikacji.}
Cz�sto okazuje si�, �e integrowane aplikacje to istniej�ce od dawna systemy,
kt�re nie do��, �e nie daj� mo�liwo�ci zmian w �r�d�ach, to dodatkowo
udost�pniaj� mocno ograniczony interfejs zewn�trzny. W takiej sytuacji
programi�ci musz� niekiedy ucieka� si� do stosowania niskopoziomowych
modyfikacji (np. na bazie danych, kt�rej u�ywa aplikacja) czy w ostateczno�ci
do dekompilacji jej modu��w.

\paragraph{Brak uniwersalnego standardu wymiany danych pomi�dzy aplikacjami.}
Pomimo, �e problemy z komunikacj� przy integracji danych znane s� od wielu lat,
du�a cz�� aplikacji nie wspiera �adnego uniwersalnego formatu danych, co
powoduje konieczno�� specjalizowania platform integracyjnych pod k�tem
pod��czanych do niej aplikacji. Jako rozwi�zanie tego problemu cz�sto stosuje
si� XML i Web serwisy. Jednak wsparcia dla ich obs�ugi mo�emy oczekiwa�
jedynie w przypadku stosunkowo nowych aplikacji. Nawet te technologie nie daj�
gwarancji pe�nej uniwersalno�ci ze wzgl�du na dodatkowe rozszerzenia, kt�re si�
wykszta�ci�y w ramach Web serwis�w (o kt�rych wi�cej informacji w dalszej
cz�ci pracy). Warto zauwa�y�, �e ten sam problem - brak wsp�pracy pomi�dzy
aplikacjami oficjalnie wspieraj�cymi pewien standard - by� g��wn� przeszkod� w
stosowaniu CORBY-y - zaawansowanego protoko�u, umo�liwiaj�cego wsp�prac�
aplikacji stworzonych w r�nych j�zykach i dzia�aj�cych na r�nych platformach.
\index{platforma integracyjna}
\paragraph{Utrzymanie platformy integracyjnej.} O ile uruchomienie platformy
integracyjnej - systemu komunikuj�cego ze sob� wiele r�norodnych aplikacji -
jest trudnym zadaniem, o tyle utrzymanie go mo�e by� jeszcze trudniejsze.
Rzadko zdarza si�, by by�a ona ca�kowicie scentralizowana i nie wymaga�a
ingerencji lub chocia� konfiguracji integrowanych aplikacji. W zwi�zku z
powy�szym, osoby, kt�re odpowiadaj� za utrzymanie takiej platformy, musz�
posiada� wiedz� na ich temat. Dodatkowo, ci�ko jest t� wiedz� utrzyma�,
zw�aszcza gdy pracownicy cz�sto si� zmieniaj�. Natura platformy
integracyjnej wymusza jej automatyzacj�, co powoduje, �e mo�e ona
bezproblemowo dzia�a� bardzo d�ugo, wr�cz przezroczy�cie dla wszystkich a� do
wyst�pienia pierwszego problemu lub potrzeby wprowadzenia modyfikacji. Wtedy
okazuje si�, �e osoby, od kt�rych wymaga si� tej wiedzy s� tymi, kt�re
t� platform� tworzyli, co z ich punktu widzenia zdecydowanie zwi�ksza nak�ad
pracy,kt�rego by nie ponie�li, gdyby systemy by�y integrowane bez punktu
centralnego, poniewa� wtedy osoby odpowiadaj�ce za nie by�yby r�wnie�
odpowiedzialne za ich integracj�. Ostatecznie okazuje si�, �e istnienie
platformy integracyjnej jest utrudnieniem, a nie u�atwieniem dla dzia�u IT.
\index{�r�d�o danych}
\paragraph{R�norodno�� �r�de� danych.} Niekiedy pojawia si� potrzeba
integrowania danych, pochodz�cych z r�nych, a czasami bardzo nietypowych
�r�de�. Przyk�ad mo�e stanowi� wypowied� pewnego biznesmena, kt�ry
przedstawia� sytuacj� w momencie, gdy pojawi�o si� zarz�dzenie, wymagaj�ce
gromadzenia w jednym miejscu wszystkich informacji, zwi�zanych z rozwojem
projektu. Naturalnym sposobem komunikacji w firmach jest poczta elektroniczna
oraz spotkania, na kt�rych powstaj� notatki. W tym momencie, przeniesienie
wymienianych t� drog� ustale�, by�o czasoch�onnym zadaniem, kt�rego efekt by� ci�ki w ocenie. 
Mo�e si� wr�cz okaza�, �e jest to wymaganie niemo�liwe do spe�nienia, je�eli
format komunikacji nie zostanie wcze�niej ustalony.

\subsection{Wyzwania biznesowe}

\paragraph{Zmiana polityki wewn�trznej przedsi�biorstwa.} 
Poprawna integracja oznacza nie tylko doprowadzenie do wsp�pracy system�w
informatycznych, ale r�wnie� departament�w w firmie. Wynika to z faktu, �e
integrowane aplikacje najcz�ciej komunikuj� automatycznie, co powoduje, �e
wprowadzenie zmian do systemu przez jeden departament mo�e mie� konsekwencje
(np. w postaci realizacji zlecenia) w drugim. Konieczne jest u�wiadomienie
u�ytkownik�w o tym fakcie, tak by integracja przynosi�a korzy�ci w postaci
uproszczenia przep�ywu danych i zaoszcz�dzenia czasu, zamiast pr�b r�cznego
kontrolowania niezale�nych do tej pory aplikacji.

\paragraph{Wp�yw integracji na przedsi�biorstwo.} 
Rozwi�zania integracyjne ��cz� ze sob� wiele aplikacji, kt�rych dzia�anie jest
kluczowe dla funkcjonowania przedsi�biorstwa. Powoduje to, �e b��dy w dzia�aniu
lub niedost�pno�� platformy integracyjnej maj� bardzo powa�ne konsekwencje,
tak�e finansowe. Z drugiej strony, poprawnie funkcjonuj�ca integracja
przyspiesza przep�yw informacji i powoduje, �e ka�dy scenariusz wsp�pracy przebiega w
daj�cych si� przewidzie� krokach, co gwarantuje powtarzalno�� i wiedz� o stanie
wsp�pracy (np. stanie zlecenia, kt�re wp�yn�o i jest obs�ugiwane). Nale�y
pami�ta�, �e utrzymywanie platformy integracyjnej wymaga kontaktu pomi�dzy
osobami opiekuj�cymi si� integrowanymi systemami, a dzia�em utrzymuj�cym
platform�. Nie mo�na dopu�ci� do niekonsultowanych wcze�niej zmian w
kt�rymkolwiek z system�w, gdy� inaczej mo�e to doprowadzi� do awarii wszystkich
integrowanych system�w.


\paragraph{R�nice semantyczne w poj�ciach.} Cho� XML rozwi�zuje wiele problem�w
technicznych, to nie stanowi on odpowiedzi na r�n� semantyk� w ramach obu
system�w. We�my za przyk�ad s�owo "konto". W jednym systemie mo�e oznacza�
numer konta bankowego. Natomiast w innym mo�e to by� wewn�trzny numer
konta w korporacji, czy te� konto u�ytkownika. Dlatego poza znalezieniem
wsp�lnych poj��, konieczne jest dok�adne zdefiniowanie ich znaczenia, aby unikn�� nieporozumie�.

\section{Problemy integracji mobilnej}

Integracja mobilna niesie za sob� wyzwania podobne do przedstawionych
w poprzednim rozdziale. Niekt�re z nich nie by�y jednak, a� tak du�ym
problemem, jakim staj� si� w kontek�cie mobilnym. Inne wynikaj� z
natury urz�dze� mobilnych i komunikacji mi�dzy nimi. 
Wszystkie te problemy nale�y wzi�� pod uwag� podczas tworzenia
rozwi�zania integracyjnego.

\subsection{R�norodno�� platform}

Pierwszym problemem, kt�ry mo�emy zauwa�y�, jest r�norodno�� mobilnych
platform. Istnieje wiele niekompatybilnych ze sob� rozwi�za� pochodz�cych od
r�nych producent�w. Je�eli spojrzymy na platformy, na kt�rych budowane s�
klasyczne systemy korporacyjne, mo�emy ograniczy� je do Javy Suna i .NET
Microsoftu. Po stronie mobilnej natomiast, mamy do wyboru wersje mobilne
wymienionych platform oraz dodatkowo Blackberry, Symbiana, Android, Palm OS oraz
kilku mniejszych dostawc�w. Opr�cz tego Apple promuje swojego iPhone, kt�ry mimo, �e
jest platform� bardzo zamkni�t�, cieszy si� wielkim zainteresowaniem w�r�d
zwyk�ych u�ytkownik�w. \\
Dodatkowym problemem jest to, �e nie istnieje tak
naprawd� standard tworzenia korporacyjnych aplikacji dla urz�dze� mobilnych,
kt�ry dostarcza�by narz�dzi u�atwiaj�cych cz�sto powtarzane czynno�ci, podobnie
jak ma to miejsce w przypadku Javy Enterprise. Tworz�c lub integruj�c
istniej�c� aplikacj� dla urz�dzenia mobilnego musimy wi�c wzi�� pod uwag�
platform� na kt�rej dzia�a, a i tak nie ob�dzie si� bez u�ycia dodatkowych
narz�dzi rozwi�zuj�cych dawno ju� rozwi�zane w klasycznych systemach problemy.\\
W przypadku urz�dze� mobilnych nie mo�emy te� skorzysta� ze sprawdzonej
tr�jwarstwowej architektury, u�ywaj�c palmtopa lub telefonu kom�rkowego, tak
jak cienkiego klienta. Jest to spowodowane niezgodno�ci� przegl�darek mobilnych
i brakiem pe�nej obs�ugi JavaScript, na kt�rym opartych jest wiele stron
internetowych. Istniej� oczywi�cie wersje stron przygotowane dla platform
mobilnych, ale daleko im do interaktywno�ci oferowanej przez bogate i
przypominaj�ce biurkowe aplikacje strony internetowe.

\subsection{Wysokie koszty}

Koszty tworzenia aplikacji mobilnych (i wynikaj�ce st�d koszty
integracji) r�ni� si� w przypadku ka�dej z platform, jednak tym co
podnosi koszt w przypadku ka�dej z nich jest:
\begin{itemize}
  \item Mniejsza spo�eczno�� programist�w tworz�cych aplikacje dla danej
  platformy. Oznacza to, �e szeroko stosowane post�powanie, polegaj�ce na
  poszukiwaniu podobnych problem�w, z kt�rymi star� si� kto� inny, nie zawsze
  przynosi r�wnie dobre rezultaty. Zwi�ksza to istotnie czas potrzebny na
  rozwi�zanie takiego problemu.
\end{itemize}

\subsection{Kana�y komunikacyjne}

\subsection{Sposoby interakcji}

\subsection{Systemy off-line}
 

%  - obecno�� wielu platform
%- wysokie koszty (zwi�zane z niezb�dnymi umiej�tno�ciami programist�w)
%- du�o wy�sza zawodno�� urz�dze� i kana��w komunikacyjnych
%- niska wydajno�� urz�dze� i kana��w komunikacyjnych
%- inny spos�b interakcji z urz�dzeniami (wynikaj�cy z ubo�szego interfejsu,
%miejsc wykorzystania oraz celu w kt�rym potrzebujemy danych) - problemy
%administracyjne - koszty infrastruktury
%- systemy off-line
\section{Istniej�ce podej�cia do integracji}
Rozwa�aj�c mo�liwe sposoby integracji technologii mobilnych z rozbudowan�
aplikacj� klasy Enteprise musimy zwr�ci� uwag� na udost�pniane z obu stron
mechanizmy.
\subsection{Transfer plik�w}
W tym przypadku dane s� zapisywane i odczytywane z plik�w. Niezb�dna jest
wsp�lna przestrze� dyskow� lub spos�b przesy�ania (np. FTP). Z tego powodu
musimy odrzuci� takie rozwi�zanie jako niedaj�ce si� zaimplementowa� w
�rodowisku mobilnym.
\subsection{Wsp�lna baza danych}
Dane s� sk�adowane we wsp�lnej bazie danych. Ze wzgl�du na zawodno�� transferu
oraz d�ugi czas przesy�ania danych pomi�dzy urz�dzeniem mobilnym a serwerem, a
tym samym konieczno�� d�ugiego oczekiwania, rezygnujemy z takiego rozwi�zania.
\subsection{RPC}
Zdalne wywo�ywanie procedur (ang. Remote Procedure Call) jest mechanizmem kt�ry
pozwala wykonywa� procedury udost�pniane przez zdalny system. Za�o�eniem
wywo�a� RPC jest synchroniczno��, kt�ra wprawdzie nie oznacza, �e musimy czeka�
na rezultat dzia�ania (mo�emy za�o�y�, �e wy��cznie zlecamy wykonanie
procedury, pobieraj�c wynik potem),ale �e zak�adamy niezawodno�� komunikacji
mi�dzy urz�dzeniami.

\subsection{Wiadomo�ci}
Komunikacja za pomoc� wiadomo�ci jest asynchroniczna, dodatkowo
dobry system przesy�ania wiadomo�ci gwarantuje jej dostarczenie do adresata,
lub informacj� o jej nie dostarczeniu. Musimy pami�ta�, �e warunki w jakich
mo�emy komunikowa� si� z urz�dzeniem przeno�nym s� dalekie od doskona�ych, a mimo to u�ytkownik oczekuje
pe�nej niezawodno�ci systemu. To sprawia, �e nale�y go tak zaprojektowa�, aby
przerwy w komunikacji by�y jak najmniej odczuwalne. Systemy komunikacyjne takie
jak RPC, RMI czy CORBA projektowane by�y dla warunk�w w kt�rych problemy
komunikacyjne s� sytuacj� nadzwyczajn�. Tymczasem w przypadku ��czno�ci
bezprzewodowej utrata po��czenia nie jest niczym czego nie mogliby�my si�
spodziewa�. W tym momencie idea przesy�ania wiadomo�ci wydaje si� skutecznym 
rozwi�zaniem tych problem�w. Podczas tworzenia przyk�adowej implementacji
napotkali�my jednak na problemy, kt�re uniemo�liwi�y wykorzystanie go do
integracji, kt�ra jest przedmiotem tej pracy. 
  
    
\chapter{Metody realizacji aplikacji mobilnej}

Przyjrzyjmy si� jednak najpierw mo�liwym sposobom zdalnego dost�pu do danych z
poziomu urz�dzenia mobilnego. Przedstawione sposoby by� mo�e nie wyczerpuj�
wszystkich mo�liwo�ci podzia�u, jednak daj� obraz tego w jaki spos�b tworzone
s� dzisiaj aplikacje mobilne. 
\section{Przegl�darka mobilna}

Urz�dzenia mobilne, w tym telefony kom�rkowe, maj� wbudowane przegl�darki
internetowe, za pomoc� kt�rych coraz cz�ciej mo�na przegl�da� wszystkie
dost�pne w Internecie strony (w przeciwie�stwie do sytuacji sprzed kilku lat,
gdy jedynie cz�� z nich, posiadaj�ca wersj� WAP, by�a dost�pna). W przeci�gu
bardzo kr�tkiego czasu dokona� si� du�y post�p w rozwoju mobilnych
przegl�darek. Ci�gle jednak nie zapewniaj� one wygody por�wnywalnej z
przegl�daniem stron na tradycyjnym komputerze. Z drugiej strony, korzystanie z
aplikacji zbudowanych w oparciu o model tr�jwarstwowy (z u�yciem tak zwanego 
cienkiego klienta - czyli przegl�darki WWW) niesie za sob� szereg korzy�ci, takich jak
brak potrzeby instalowania aplikacji, �atwo�� wdro�enia czy wreszcie du�a
przeno�no�� pomi�dzy r�nymi platformami.

\subsection{Rodzaje przegl�darek mobilnych}

Omawiaj�c temat przegl�darek mobilnych nale�y zwr�ci� uwag� na to, �e na rynku
znajduje si� obecnie wiele tego typu produkt�w. Nie wszystkie s� bezpo�redni�
konkurencj�. Cz�� z nich mo�e dzia�a� tylko w obr�bie platformy na kt�r�
zosta�y zaprojektowane. Szczeg�lnie widoczne jest to w wypadku przegl�darki
Safari, kt�ra jest u�ywa praktycznie tylko na urz�dzeniach mobilnych firmy
Apple. Nale�y bra� to pod uwag� przy projektowaniu strony mobilnej. Mo�e
si� okaza�, �e mo�na ograniczy� wysi�ki zapewnienia pe�nej zgodno�ci na
wszystkich mo�liwych przegl�darkach, poniewa� w obr�bie organizacji funkcjonuj�
urz�dzenia tylko jednej marki (na przyk�ad Blackberry).


\subsubsection{Uproszczone przegl�darki mobilne}

Pierwszym rodzajem przegl�darek spotykanych na urz�dzeniach mobilnych s�
przegl�darki wy�wietlaj�ce uproszczon� wersj� stron internetowych.
Dlatego przeniesienie istniej�cych portali korporacyjnych pod platform�, kt�ra
ich u�ywa wi��e si� z dodatkow� prac� wynikaj�c� z konieczno�ci przystosowania
sposobu wy�wietlania portali do mo�liwo�ci przegl�darki.


\paragraph{Opera Mini}

Jest to najpopularniejsza przegl�darka mobilna dzia�aj�ca w oparciu o platform�
Java 2 Micro Edition. Charakterystyczn� cech� tej przegl�darki jest serwer
proxy, kt�ry optymalizuje zawarto�� stron przed za�adowaniem. Wad� tego
rozwi�zania mo�e by� niech�� do u�ywania go w przedsi�biorstwach w celu
uzyskiwania dost�pu do wra�liwych danych. Strach przed ich utrat� mo�e by� du�o
wi�kszy ni� korzy�ci jakie mo�na odnie�� z u�ywania tego programu.

\begin{center}
\setlength\fboxsep{5pt}  
\setlength\fboxrule{0.0pt}
\fbox{\scalebox{0.5}{\includegraphics{img/opera_onet.png} }}
\end{center}

\paragraph{Blackberry Browser}

Blackberry Browser jest domy�ln� aplikacj� pozwalaj�c� na przegl�danie zasob�w
intranetu lub internetu. Podstawowym trybem dzia�ania tej przegl�darki jest
umo�liwianie dost�pu do wewn�trznych zasob�w firmy. Wi��e si� to bezpo�rednio
ze sposobem dzia�ania urz�dze� Blackberry opartym o sie� wirtualn�. Ka�de
urz�dzenie mobilne jest tunelowane przez powszechnie dost�pn� sie�
telekomunikacyjn� do sieci wewn�trznej firmy. Tunelowanie zajmuje si�
dzia�aj�cy w obr�bie firmowej sieci serwer BES (Blackberry Enterprise Server).
Jest on jedynym elementem wystawionym na zewn�trz i poprzez szyfrowanie
po��czenie umo�liwia przekierowywanie pakiet�w z i do urz�dze� mobilnych.
Dzi�ki temu ka�dy u�ytkownik ma zapewniony bezpieczny dost�p do informacji
firmowych. Ma to ogromne znaczenie przy wyborze sposobu umobilniania. W
przypadku pozosta�ych przegl�darek mobilnych jedn� gwarancj� bezpiecze�stwa
jest http przy czym nie ma �adnego zapewnienia bezpiecze�stwa danych, kt�re
przechodz� przez serwery proxy (tak jak w wypadku przegl�darki Opera Mini).
Dlatego w wypadku, gdy g��wnym czynnikiem dycyduj�cym o wyborze jest
bezpiecze�stwo warto rozwa�y� zastosowanie rozwi�za�, kt�re je gwarantuj�.

\begin{center}
\setlength\fboxsep{5pt}  
\setlength\fboxrule{0.0pt}
\fbox{\scalebox{0.2}{\includegraphics{img/7290.jpg} }}
\end{center}

\subsubsection{Pe�ne przegl�darki mobilne}

\paragraph{Opera Mobile}

Opera Mobile jest jedn� z najpopularniejszych pe�nowarto�ciowych
przegl�darek mobilnych. Zosta�a zbudowana na bazie silnika przegl�darki
desktopowej. Zachowuje prawie ca�kowit� zgodno�� z pe�n� wersj�. Co oznacza, �e
w �rodowisku opartym tylko i wy��cznie o t� mark� mo�liwe jest u�ywanie stron
internetowych bez �adnych zmian przystosowuj�cych je do wersji mobilnej.
Niestety ze wzgl�du na niewielk� popularno�� przegl�darki desktopowej oraz jej
ca�kowity brak zgodno�ci z dwoma dominuj�cymi przegl�darkami (wynikaj�cy z
bardzo restrykcyjnego pilnowania standard�w przez deweloper�w Opery), zwykle
niezb�dne s� zmiany w konstrukcji strony. Dzia�a na urz�dzeniach korzytaj�cych
z platform Symbian S60 oraz Windows Mobile.

\begin{center}
\setlength\fboxsep{5pt}  
\setlength\fboxrule{0.0pt}
\fbox{\scalebox{0.4}{\includegraphics{img/opera_mobile.png} }}
\end{center}

\paragraph{Internet Explorer Mobile}

Poniewa� firma Microsoft posiada od dawna systemy operacyjne przeznaczone na
platformy mobilne (Windows Mobile oraz starszy Windows CE) razem z nimi zawsze
dostarczana jest przegl�darka internetowa. Jest sporo wolniejsza od
konkurencji, ale ze wzgl�du na fakt, �e jest preinstalowana na platformach
cieszy si� du�ym powodzeniem w�r�d mobilnych u�ytkownik�w. Przy wdra�aniu
rozwi�za� mobilnych w firmie, w kt�rej u�ywane s� urz�dzenia typu Pocket PC,
nale�y si� spodziewa�, �e zostanie ona narzucona z g�ry jako klient naszej
mobilnej strony internetowej. Pozwala na dzia�anie zar�wno w trybie mobilnym,
odczytuj�cym uproszczony zapis html, jak i na prac� z pe�nymi wersjami stron
internetowych. Niestety silnik tej przegl�darki zosta� napisany zupe�nie od
nowa przez co nie jest w pe�ni zgodny z tym, kt�ry jest u�ywany w pe�nej wersji
Internet Explorera. Mo�e to zwi�kszy� nak�ad pracy niezb�dny do uzyskania
mobilnej wersji naszej strony. 

\begin{center}
\setlength\fboxsep{5pt}  
\setlength\fboxrule{0.0pt}
\fbox{\scalebox{0.4}{\includegraphics{img/ie_mobile.png} }}
\end{center}

\paragraph{Safari}

Przegl�darka firmy Apple u�ywana g��wnie w jej produktach. Jej mobilna
wersja pojawi�a si� na rynku w momencie wprowadzenia telefonu iPhone.
Charakteryzuje si� bardzo dobr� obs�ug� JavaScripta oraz pe�n� zgodno�ci� z jej
wersj� desktopow�. W odr�nieniu od produkt�w Opera Software na urz�dzeniach
stacjonarnych firmy Apple jest one przegl�dark� dominuj�c�, wi�c dzi�ki pe�nej
zgodno�ci nak�ady pracy potrzebne na przystosowanie oraz wdro�enie s� znacznie
mniejsze ni� w przypadku konkurencji.

\begin{center}
\setlength\fboxsep{5pt}  
\setlength\fboxrule{0.0pt}
\fbox{\scalebox{0.4}{\includegraphics{img/iphone_screen.jpg} }}
\end{center}

\subsection{Zagadnienia zwi�zane z projektowaniem stron}

W poprzednich rozdzia�ach zosta�o przedstawionych kilka najpopularniejszych
przegl�darke mobilnych. Wyr�nione zosta�y pewne elementy, kt�re wymuszaj�
zastosowanie specjalnej metodologii przy przystosowywaniu stron internetowych.
Podzia� na przegl�darki uproszczone i pe�ne przegl�darki mobilne sugeruje dwa
r�ne podej�cia do tego zagadnienia. Pierwszy spos�b przystosowania stron
internetowych zak�ada, �e mamy do czynienia z pe�nymi przegl�darkami. Poniewa�
r�ni� si� one g��wnie brakiem wsparcia dla niekt�rych element�w Javascript, to
praca zwi�zana z ich umobilnianiem wi��e si� g��wnie z upraszczeniem pewnych
element�w lub przepisywaniem ich tak by by�y kompatybilne. Zwykle trzeba
zrezygnowa� te� z pewnych zaawansowanych funkcjonalno�ci, kt�re dzia�aj� w
oparciu o technologi� Javascript. Je�li projektujemy portal od pocz�tku mo�emy
w bardzo prosty spos�b przestrzegaj�c kilku podstawowych regu� umo�liwi� dost�p
do niego z urz�dze� mobilnych. Najwa�niejsze aspekty, kt�re nale�y mie� na
uwadze:
\begin{itemize}
  \item Nale�y pami�ta� o budowaniu prostych i funkcjonalnych uk�ad�w stron.
  Dotyczy to w szczeg�lno�ci form, kt�re powinny mie�ci� si� na jednym ekranie.
  Co wi�cej nale�y wykorzystywa� listy opcji tam gdzie tylko si� da, tak by
  u�ytkonik musia� w minimalnym stopniu u�ywa� klawiatury.
  \item Nie mo�na zak�ada�, �e do nawigacji b�dzie wykorzystywana wirtualna
  myszka. Nale�y tak projektowa� strony, by mo�na by�o je obs�ugiwa� innymi
  rodzajami kontroler�w (na przyk�ad przy pomocy technologii trackwheel).
  Wprowadzanie danych z klawiatury urz�dzenia mobilnego jest bardzo niewygodne,
  wi�c nale�y zadba� o to by u�ytkownik nie musia� wprwadza� du�ych ilo�ci
  danych(wspomniane wcze�niej listy opcji rozwi�zauj� w pewnym stopniu ten
  probem)
  \item Urz�dzenia mobilne dysponuj� innymi, cz�sto znacznie mniejszymi
  rozdzielczo�ciami ni� systemy stacjonarne. Niekt�re przegl�darki takie jak
  Opera Mobile oferuj� funkcje szybkiej nawigacji w obr�bie du�ych stron,
  dzi�ki czemu problem ten jest w pewien spos�b ograniczany. Je�li jednak chcemy
  dostarczy� nalepsz� mo�liw� obs�ug� stron nale�y wykorzysta� pewne elementy
  CSS, takie jak zapytania o typ urz�dzenia, by dostosowa� stron� do konkretnej
  rozdzielczo�ci. Dobrym przyk�adem jak wa�ne jest uwzgl�dnienie tych
  rozbie�no��i jest Opera Desktop oraz Opera Mini. Na rysunku \ref{fig:omini}
  wida� przyk�adow� realizacj� strony z wykorzystaniem specjalnych styl�w dla
  przegl�darki moblinej.
  \begin{figure}[htb]
    \begin{center}
    \includegraphics[angle=0,scale=.6]{img/desktopmobile.png}
    \end{center}
    \caption{Ta sama strona otwarta w Opera Desktop 9.5 oraz Opera Mini 4.1}
    \label{fig:omini}
\end{figure} 

  \item  Nale�y r�wnie� wzi�� pod uwage r�nice pomi�dzy przegl�darkami
 mobilnymi - w szczeg�lno�ci oferowanymi przez r�nych producent�w. R�nice w
 silnikach tych przegl�darek bywaj� tak du�e, �e cz�sto projektant jest
 zmuszony do przygotowywania r�nych wersji tej samej strony dla r�nych
 urz�dze� mobilnych. Niezgodno�ci te wynikaj� g��wnie z nie przestrzegania
 standard�w przez producent�w, oraz z wprowadzania do swoich rozwi�za� dodatkowych funkcjonalno�ci nie spotykanych na innych platformach. Dobrym przyk�adem takich dzia�a� jest wprowadzenie w najnowszej wersji mobilnej przegl�darki
	Safari na urz�dzeniu iPhone mo�liwo�ci budowaniu animacji poklatkowych przy
	pomocy samych styl�w CSS. Jest to bardzo ciekawa i przydatna funkcjonalno��,
	ale ca�kowicie nie przeno�na na inne przegl�darki.

\end{itemize}
 
\section{Generatory aplikacji}
\index{Generatory aplikacji}
Wielu producent�w podejmuje pr�by zbudowania �rodowisk, umo�liwiaj�cych szybkie
tworzenie aplikacji mobilnych (tak zwane RAD - Rapid Application Development).
Maj� one ��czy� zalety obu przedstawionych wcze�niej �wiat�w :

\begin{itemize} 
\item{Pozwala� na �atwe i szybkie tworzenie aplikacji dost�powych - tak jak
mobilne przegl�darki}
\item{Dostarcza� elastyczno�ci i rozbudowanych interfejs�w u�ytkownika - tak
jak aplikacje mobilne}
\end{itemize}

W dalszej cz�ci niniejszej pracy przedstawione zostan� dwa tego typu
rozwi�zania, pozwalaj�ce na oszcz�dzenie znacznej ilo�ci pracy przy budowaniu
nowych aplikacji.

\index{Blackberry}
\subsubsection{Generator dedykowany - Blackberry MDS}
\index{Blackberry MDS}
Blackberry MDS jest przyk�adem narz�dzia wyspecjalizowanego do tworzenia
aplikacji tylko i wy��cznie na platform� jednego producenta. Rozwi�zanie to
opiera si� na integracji ma�ej aplikacji zainstalowanej na urz�dzeniach
klienckich (MDS Runtime) z us�ug� typu Web Service dzia�aj�c� na serwerze. 
Dzi�ki niemu mo�liwe jest tworzenie aplikacji, u�ywaj�c podej�cia opisowego.
Zamiast pisa� ca�� aplikacj�, u�ywaj�c j�zyka programowania wraz z
bibliotekami, oferuj�cymi interfejs u�ytkownika, mo�emy przy u�yciu graficznego
edytora projektowa� takie elementy, jak ekrany czy pola z danymi. Wszystko to
przy u�yciu standardowego drag and drop. Nast�pnie, mo�emy po��czy� tak
zbudowane komponenty przy u�yciu r�nych przewodnik�w(wizards), czy te�
edytor�w. Co wi�cej, aplikacja MDS daje nam pe�n� kontrol� nad przep�ywem
sterowania - przy pomocy j�zyka JavaScript jeste�my w stanie zaimplementowa� niemal dowoln�
logik� aplikacji. Aplikacje typu enterprise oraz �r�d�a danych dost�pne s� dla
aplikacji mobilnej poprzez standardow� technologi� Web Services.

\begin{figure}[htb]
    \begin{center}
    \includegraphics[angle=0,scale=0.7]{img/mds1.png}
    \end{center}
    \caption{Architektura aplikacji opartej o BlackBerry MDS}
    \label{fig:mds1}
\end{figure} 

\index{BES}
\index{Blackberry Enterprise Server}

Aplikacja MDS dla Blackberry jest to aplikacja typu rich-client, zbudowana na
podstawie metadanych zdefiniowanych w j�zyku XML. Te metadane wraz z zasobami,
takimi jak obrazki, gromadzone s� w pakietach, kt�re nast�pnie publikowane s� na
serwerze BES. Serwer ten przy u�yciu technologii push rozsy�a aplikacje do
terminali mobilnych Blackberry. Dzi�ki scentralizowanej metodzie
dystrybucji oraz zarz�dzania aplikacjami mobilnymi, koszty utrzymania mobilnej
infrastruktury s� znacznie ni�sze.

\begin{figure}[htb]
    \begin{center}
    \includegraphics[angle=0,scale=0.7]{img/mds2.png}
    \end{center}
    \caption{Spos�b zarz�dzania aplikacjami opartymi o MDS}
    \label{fig:mds2}
\end{figure} 


\paragraph{Us�uga Blackberry MDS}
Us�uga MDS Blackberry zosta�a zaprojektowana tak, by umo�liwi� wymian� danych
pomi�dzy urz�dzeniami Blackberry, a aplikacjami typu enterprise. Zapewnia ona:
\begin{itemize} 
\item{Izolacj� platformy od szczeg��w, zwi�zanych z integracj� �r�de� danych}
\item{Standard asynchronicznej wymiany komunikat�w}
\item{Przezroczyste dla u�ytkownika i developera protoko�y bezpiecze�stwa}
\item {Transformacje wiadomo�ci}
\item {Wspomnian� wcze�niej scentralizowan� administracj� przy pomocy
technologii push}
\end{itemize}

\paragraph{Blackberry MDS runtime}
\index{MDS runtime}
Po stronie urz�dzenia mobilnego zainstalowany jest kontener aplikacji MDS.
Zajmuje si� on instalacj�, wyszukiwaniem oraz przegl�daniem aplikacji MDS.
Dostarcza r�wnie� niezb�dne funkcjonalno�ci do tych aplikacji, takie jak
interfejs u�ytkownika, kontener danych oraz us�ugi komunikacyjne klient-serwer.
\index{Netbeans}
\subsubsection{Generator uniwersalny - Netbeans Mobile IDE}

Drugim przyk�adem generatora aplikacji mobilnych jest Netbeans Mobility Pack
oraz zwi�zane z nim �rodowisko programistyczne Netbeans. W przeciwie�stwie do
poprzedniego przyk�adu, to �rodowisko pozwala budowa� uniwersalne aplikacje,
kt�re dzia�aj� na niemal�e dowolnej platformie mobilnej wspieraj�cej J2ME.
Rozpoczynaj�c tworzenie aplikacji powinni�my zwr�ci� uwag� ba udogodnienia
przygotowane przez tw�rc�w Netbeans. Zamiast pisa� setki linijek kodu Java,
mo�emy u�y� specjalnych narz�dzi generuj�cych go dla nas. Do konfigurowania
preferowanego przez nas zachowania aplikacji s�u�� interaktywne diagramy
przep�ywu sterowania pomi�dzy ekranami aplikacji oraz graficzne edytory
interfejsu u�ytkownika, pozwalaj�ce na budowanie z gotowych komponent�w
ekran�w naszej mobilnej aplikacji. Poni�ej przedstawiony jest przyk�ad diagramu
przep�ywu sterowania aplikacji typu Hello World.


\begin{figure}[htb]
    \begin{center}
    \includegraphics[angle=0,scale=0.4]{img/netbeans_layout.png}
    \end{center}
    \caption{�rodowisko NetBeans z diagramem przep�ywu}
    \label{fig:netbeans_layout}
\end{figure} 


Przyk�adowy diagram przep�ywu sk�ada si� z dw�ch element�w - bloku,
reprezentuj�cego stan wyj�ciowy aplikacji mobilnej (a jednocze�nie stan ko�cowy)
oraz bloku, reprezentuj�cego g��wny i jedyny ekran. Elementy te po��czone s�
dwiema strza�kami, kt�re przedstawiaj� przep�yw sterowania. Strza�ka skierowna w
stron� formy ma sw�j pocz�tek w zdarzeniu Started. Jest to zdarzenie wywo�ywane
w momencie uruchomienia aplikacji. Dzi�ki takiej konfiguracji, natychmiast po
inicjalizacji aplikacja przechodzi do ekranu g��wnego. Strza�ka w drug� stron�
podpi�ta jest do komendy exitCommand. Wyzwala ona zamkni�cie aplikacji (jej
drugi koniec wskazuje blok wyj�ciowy).
Ekran g��wny zosta� zaprojektowany tak, by wy�wietli� napis oraz komend�,
pozwalajac� na opuszczenie aplikacji.

\begin{figure}[htb]
    \begin{center}
    \includegraphics[angle=0,scale=0.5]{img/netbeans_design.png}
    \end{center}
    \caption{Tworzenie interfejsu u�ytkownika w �rodowisku NetBeans}
    \label{fig:netbeans_design}
\end{figure} 



Tak jak wspomniano wcze�niej, tego typu ekrany buduje si� z gotowych
komponent�w. �rodowisko Netbeans udost�pnia zestaw predefiniowanych komponent�w
o podstawowej funkcjonalno�ci. U�ytkownik mo�e odnale�� je na palecie bocznej
programu. W wersji 6.5 paleta ta wygl�da tak, jak na poni�szym rysunku.

\begin{figure}[htb]
    \begin{center}
    \includegraphics[angle=0,scale=0.5]{img/netbeans_palette.png}
    \end{center}
    \caption{Paleta komponent�w w �rodowisku NetBeans}
    \label{fig:netbeans_palette}
\end{figure} 


Komponentowe budowanie aplikacji znacznie przyspiesza proces ich tworzenia. Co
wi�cej, w razie potrzeby istnieje mo�liwo�� definiowania w�asnych komponent�w,
kt�re p�niej mo�na wykorzystywa� w wielu aplikacjach mobilnych. Miejsce na
w�asne komponenty mo�na zobaczy� w zak�adce custom na standardowej palecie
komponent�w.
Dzi�ki zastosowaniu przedstawionych wcze�niej edytor�w, uda�o si� zbudowa�
bardzo prost� aplikacj� zdoln� do wy�wietlenia napisu. Zosta�o to zrobione bez
napisania linijki kodu w j�zyku Java, u�ywaj�c jedynie narz�dzi graficznych.
Tak zbudowana aplikacja dzia�a na praktycznie ka�dej konfiguracji mobilnej i
gwarantuje jednakowe zachowanie pomi�dzy r�nymi platformami. 

\begin{figure}[htb]
    \begin{center}
    \includegraphics[angle=0,scale=0.5]{img/netbeans_simulator.png}
    \end{center}
    \caption{Emulator telefonu dostarczany przez Suna}
    \label{fig:netbeans_simulator}
\end{figure}

Do konstruowania bardziej zaawansowanych aplikacji nale�y pos�u�y� si�
j�zykiem Java. Dost�p do kodu �r�d�owego automatycznie generowanej aplikacji
pozwala na elastyczne modyfikowanie jej zachowania w zakresie du�o szerszym ni� pozwala na
to jakakolwiek graficzna konfiguracja. Co wi�cej, ka�dy element, kt�ry zosta�
przez nas skonfigurowany w jednym ze wspomnianych wcze�niej edytor�w jest
powi�zany z wygenerowanym kodem w j�zyku Java, z kt�rym mo�na si� zapozna� w
zak�adce source.

\begin{figure}[htb]
    \begin{center}
    \includegraphics[angle=0,scale=0.5]{img/netbeans_source.png}
    \end{center}
    \caption{Edycja kodu �r�d�owego w NetBeans}
    \label{fig:netbeans_source}
\end{figure}


Wyszarzone obszary to kod, kt�ry zosta� automatycznie wygenerowany przez
�rodowisko Netbeans. W pozosta�ych obszarach mo�na wprowadza� w�asny kod w
j�zyku Java, kt�ry mo�e kontrolowa� dowolne aspekty zachowania si� aplikacji
mobilnej.

Przedstawione �rodowisko programistyczne jest jednym z popularniejszych podej��
do problemu szybkiego konstruowania aplikacji mobilnych. Jest elastyczne i
efektywne, dlatego pos�u�ymy si� nim przy budowie przyk�adowego systemu. 
 
\section{Rozwi�zania dedykowane}

Innym podej�ciem, przypominaj�cym aplikacje, zbudowane w oparciu o model
dwuwarstwowy, jest stworzenie aplikacji mobilnej, kt�ra samodzielnie ��czy si�
z Internetem i umo�liwia wy�wietlanie i modyfikacj� danych w spos�b
przewidziany przez jej tw�rc�w. Ogranicza to, oczywi�cie, funkcjonalno�� takiej
aplikacji do zakresu, przewidzianego w danej wersji, a wszelkie zmiany wymagaj�
jej aktualizacji. W zamian za to, otrzymujemy wi�ksz� szybko�� dzia�ania,
mo�liwo�� walidacji danych po stronie klienta oraz elastyczno�� w tworzeniu
interfejsu (kt�ry w tym przypadku nie jest ograniczony do kontrolek,
zapewnianych przez przegl�dark� WWW).

Wad�, z kt�rej niewiele os�b zdaje sobie
spraw�, jest mniejsza przeno�no�� aplikacji, nawet je�li by�yby one napisane w
j�zyku, projektowanym z my�l� o niezale�no�ci od platformy, takim jak Java.
Wynika ona z r�norodno�ci urz�dze� dost�pnych na rynku. Ca�kowita przeno�no�� mo�liwa
jest do uzyskania tylko dla prostych aplikacji, nie korzystaj�cych z
mo�liwo�ci, oferowanych przez konkretne modele urz�dze� mobilnych. Przy pr�bie
zbudowania czegokolwiek wi�kszego, natychmiast natkniemy si� na niewidzialny
mur, generowany przez ogromne r�nice pomi�dzy dost�pno�ci� profili na
poszczeg�lnych platformach oraz przez r�nice w implementacji pewnych cz�ci
samych maszyn wirtualnych. Co wi�cej, opisana tu sytuacja wskazuje na problem,
kt�ry generuje stworzona z my�l� o przeno�no�ci platforma J2ME. Jeszcze wi�ksze
problemy pojawiaj� si�, gdy chcemy rozszerzy� zasi�g naszej aplikacji o kolejne
segmenty rynku, takie jak u�ytkownicy system�w Symbian, czy Windows Mobile. W
tym momencie okazuje si�, �e zostaniemy zmuszeni do napisania nie jednej
aplikacji, ale wielu, kt�re cz�sto mog� nie mie� prawie �adnych cz�ci
wsp�lnych. 

\chapter{Hybrydowe podej�cie do integracji mobilnej}

W poprzednim rozdziale zaprezentowali�my sposoby tworzenia mobilnych aplikacji.
W ramach niniejszej pracy chcemy zaprezentowa� nowe i~nie stosowane dot�d
podej�cie. Po��czyli�my w nim zalety zapewniane przez aplikacj� zbudowan� w
oparciu o model tr�jwarstwowy, z tymi typowymi dla samodzielnej aplikacji. 
W cz�ci implementacyjnej naszej pracy  stworzyli�my szablon, u�atwiaj�cy
tworzenie aplikacji, posiadaj�cych mobilny interfejs, 
z poziomu kt�rego mo�na sterowa� aplikacj� serwerow�. 
Stworzone rozwi�zanie stanowi raczej pomys�. Mo�na go rozwija� i
wykorzystywa� do budowy bardziej skomplikowanych system�w. 
\newline
Szablon sk�ada si� z dw�ch cz�ci - mobilnej i~serwerowej.
U�ytkownik mo�e okre�li� wygl�d ekranu przesy�anego do urz�dzenia mobilnego za
pomoc� j�zyka XML. Po stronie mobilnej kod XML jest interpretowany do postaci
element�w interfejsu klienta, na przyk�ad p�l tekstowych, wyboru czy rozwijanych list. 
Rozszerza to mo�liwo�ci aplikacji w stosunku do zwyk�ej strony internetowej o
elementy niedost�pne w przypadku statycznej strony WWW, jak rysunki
wektorowe, obs�uga przycisk�w urz�dzenia, czy mo�liwo�� wy�wietlania strony
przez jaki� czas.

W typowym przypadku scenariusz u�ycia aplikacji b�dzie wygl�da� nast�puj�co:

\begin{figure}[htb]
    \begin{center}
    \includegraphics[angle=0,scale=0.5]{img/pool_schema_uml.png}
    \end{center}
    \caption{Diagram przep�ywu sterowania w szablonie integracyjnym}
    \label{fig:pool_schema_uml}
\end{figure}


\begin{itemize}
  \item Tworzymy nowy 'ekran' na serwerze. Oznacza to okre�lenie wygl�du i
  zachowania aplikacji w formacie XML. Okre�lamy dost�pne dla
  danego ekranu akcje. 
  \item Aplikacja mobilna ��czy si� z serwerem i~pobiera przygotowany dla niej
  ekran. U�ytkownik wykonuje jedn� z dost�pnych na nim akcji, a informacja o
  tym jest przesy�ana do serwera. 
  \item Aplikacja mobilna pobiera kolejny ekran przygotowany przez serwer i
  dalej post�puje wed�ug powy�szego schematu. W zwi�zku z tym, �e to w�a�nie
  serwer jest w ca�o�ci odpowiedzialny za przygotowanie ekranu, to pojawia si�
  mo�liwo�� kontrolowania ekranu i~jego zachowania po stronie klienta w dowolny spos�b.
\end{itemize}


\section{Mo�liwo�ci}
Aplikacja oparta o tak zaprojektowany model posiada�aby cechy typowe dla modelu
dwuwarstwowego - w tym szybki czas reakcji, niezale�no�� wygl�du aplikacji od
stosowanego klienta oraz mo�liwo�� przeprowadzenia pewnych dodatkowych operacji
(jak walidacja) po stronie klienta. W idealnym przypadku m�g�by by�
dynamicznie przesy�any kod wykonywany po stronie klienta (odpowiednik JavaScript
w przypadku przegl�darki internetowej). Pozbawiona by�aby te� typowych wad obu rozwi�za� -
brak konieczno�ci aktualizowania aplikacji przy drobnych poprawkach (poza
b��dami w zainstalowanym oprogramowaniu). Nie by�aby r�wnie� ograniczona do
mechanizm�w oferowanych przez statyczne strony WWW - mo�na by by�o
wy�wietla� dowolne ekrany, a na nich wykonywa� dowolne akcje.
Dodatkowo, poniewa� stan aplikacji przechowywany by�by na serwerze, mo�na w
dowolnym momencie zmieni� urz�dzenie mobilne, wy��czy� aplikacj�, a potem
powr�ci� do zapami�tanego wcze�niej miejsca.
 
\paragraph{Nowe podej�cie}
 
W przypadku tworzenia aplikacji mobilnych naturalna jest ch�� przeniesienia 
funkcjonalno�ci dost�pnych w aplikacjach 'biurkowych' na urz�dzenie mobilne.
Nie zawsze jest to podej�cie s�uszne - u�ytkownik korzystaj�cy z urz�dzenia
mobilngo - palmtopa czy telefonu kom�rkowego ma znacznie ograniczone mo�liwo�ci
dzia�ania i tym co si� w takiej sytuacji liczy jest szybko�� i �atwo�� obs�ugi
aplikacji, kt�r� mo�na postawi� przed jej du�ymi mo�liwo�ciami. Dlatego naszym
zdaniem, celem tworzenia aplikacji mobilnych nie powinno by� dostarczenie
wszystkich mo�liwych funkcjonalno�ci, ale raczej maksymalna ergonomia i
prostota u�ycia, kt�re przy�wieca�y nam w trakcie tworzenia szablonu.
\newline 
Ponadto, techniczne rozwi�zania przyj�te w szablonie nie by�y - zgodnie z nasz�
wiedz� - do tej pory stosowane. Nasz szablon opiera si� na bibliotece KUIX,
kt�ra pozwala na wykorzystanie formatu XML do stworzenia interfejsu
u�ytkownika. Pozwala to na dynamiczne �adowanie komponent�w wcze�niej
przygotowywanych na serwerze.
Naturaln� drog� rozwoju by�oby rozbudowanie szablonu o mo�liwo�� dynamicznego
�adowania kodu, mechanizm�w walidacji czy tworzenia grafiki.

\subsection{Zastosowania}
Opisywana idea u�atwia tworzenie okre�lonego typu aplikacji i tak jak ka�dy
szablon nie pokrywa wszystkich istniej�cych rozwi�za�. Nasz szablon u�atwi
pisanie aplikacji, kt�re nie wymagaj� przesy�ania du�ych ilo�ci danych i od
kt�rej oczekujemy interaktywno�ci wi�kszej ni� w przypadku strony internetowej
pozbawionej JavaScript. 

\paragraph{Ankietowanie}
Jednym z zastosowa� kt�re mo�emy sobie wyobrazi� jest zdalne ankietowanie
u�ytkownik�w. Stworzenie aplikacji kt�ra na ekranie wy�wietla list� dost�pnych
odpowiedzi i pozwala na wybranie jednej z nich, jest przy u�yciu
przedstawianego szablonu prostym zadaniem.
\paragraph{Zdalne podejmowanie decyzji}
Podobnie jak powy�ej, u�ytkownicy mog� wybra� jedn� z mo�liwych decyzji, w
takim przypadku potrzebne b�dzie rozbudowanie szablonu o zaawansowane
mechanizmy bezpiecze�stwa, jednak idea pozostaje podobna.
\paragraph{Zdalne wy�wietlanie tre�ci}
Urz�dzenie mobilne mo�e by� te� jedynie miejscem kt�re wy�wietla dostarczane do
niego informacje. Interakacja z u�ytkownikiem mo�e by� w tym momencie
wy��czona, a urz�dzenie mobilne mo�e pe�ni� rol� zdalnie kontrolowanego ekranu.

\section{Architektura}

\subsection{Aplikacja serwerowa}

\subsection{Aplikacja mobilna}
 


\subsection{Aplikacja serwerowa}

Przedstawimy teraz architektur� cz�ci serwerowej, komponenty, z kt�rych si�
sk�ada oraz wykorzystywane technologie.
\newline
Aby zrozumie� projekt ca�ego systemu, wprowad�my pewne poj�cia w naszym
systemie. Wynikaj� z nich dalsze rozwi�zania: 
\begin{itemize}
  \item \emph{Ankieta} - tre��, dla kt�rej istniej� dost�pne zdefiniowane
  odpowiedzi.
 \item \emph{Odpowied�} - jedna z mo�liwo�ci wyboru dost�pnych dla danej ankiety
  \item \emph{U�ytkownik} - konto, z kt�rym zwi�zane s� uprawnienia i~udzielone
  w ankietach odpowiedzi. Jeden u�ytkownik mo�e udzieli� wielu odpowiedzi w
  trakcie dzia�ania ca�ego systemu.
\end{itemize}


Aplikacja serwerowa zbudowana zosta�a w oparciu o model tr�jwarstwowy, z
podw�jnym dost�pem (widokiem) do danych. 



\begin{figure}[htb]
    \begin{center}
    \includegraphics[angle=0,scale=0.7]{img/diag2.png}
    \end{center}
    \caption{Architektura cz�ci serwerowej szablonu integracyjnego}
    \label{fig:pool_arch2}
\end{figure}

Przedstawmy teraz poszczeg�lne warstwy systemu wraz z przyj�tymi w nich
za�o�eniami.

\paragraph{Serwer}
Aplikacja serwerowa jest uruchamiana na kontenerze Apache Tomcat 6.0.

\paragraph{Baza danych}
W oparciu o przedstawione wcze�niej za�o�enia stworzony zosta� schemat
bazy danych widoczny na rysunku \ref{fig:pool_db}.


\begin{figure}[htb]
    \begin{center}
    \includegraphics[angle=0,scale=0.7]{img/pool_schema_o.png}
    \end{center}
    \caption{Schemat koncepcyjny bazy danych cz�ci serwerowej szablonu
    integracyjnego}
    \label{fig:pool_db}
\end{figure}

\begin{figure}[htb]
    \begin{center}
    \includegraphics[angle=0,scale=0.7]{img/pool_schema.png}
    \end{center}
    \caption{Schemat fizyczny bazy danych cz�ci serwerowej szablonu
    integracyjnego}
    \label{fig:pool_db2}
\end{figure}


Schemat koncepcyjny(rysunek \ref{fig:pool_db}) jest tylko przyk�adem. Stanowi
podstaw� do prezentacji mo�liwo�ci szkieltu Kuix. Konstrukcja tego schematu mo�e
by� dowolna. Kluczowym zagadnieniem jest transformacja danych zawartych w systemie
na XML reprezentuj�cy interfejs widoczny na urz�dzeniu. Kuix jest na
tyle elastyczny, �e mo�liwe jest pod��czenie do niego dowolnego �r�d�a danych
generuj�cego XML (w szczeg�lno�ci danych nie musi dostarcza� baza relacyjna -
mo�na na przyk�ad pod��czy� baz� obiektow� Lotus Notes).
\newline
Na postawie schematu koncepcyjnego stworzony zosta� schemat fizyczny bazy
danych widoczny na rysunku \ref{fig:pool_db2}.
\newline


\index{Hibernate}
\paragraph{Hibernate}
Dost�p do danych realizowany jest za pomoc� mapowania relacyjno-obiektowego
Hibernate. Teoretycznie pozwala to, a na pewno znacznie u�atwia, ewentualn�
zmian� dostawcy bazy danych. Dodatkowo uwalnia nas z konieczno�ci r�cznego
pisania zapyta� SQL. 

\index{Spring}
\paragraph{Spring Framework}
Spring jest zestawem narz�dzi u�atwiaj�cym tworzenie aplikacji. Pozwala on
mi�dzy innymi na deklaratywne zarz�dzanie transakcjami, zarz�dzanie
cyklem �ycia obiekt�w oraz zwalnia programist� z konieczno�ci pisania
 cz�sto powtarzanego kodu przez dostarczenie odpowiednich szablon�w. Dodatkowo,
 wymusza on pewien model programowania i~sprawia, �e pisane aplikacje s�
 bardziej przejrzyste i~�atwiejsze w utrzymaniu. Jak ju� zosta�o wspomniane,
 aplikacja serwerowa jest aplikacj� typu Web, z kt�r� to framework Spring
 doskonale si� integruje.
 
\index{JSF}
\paragraph{Apache Myfaces} 
Myfaces jest implementacj� specyfikacji Java Server Faces pozwalaj�c� na
wygodne tworzenie widoku aplikacji. W naszym przypadku interfejsem b�dzie
dynamicznie generowana strona WWW, cho� specyfikacja JSF nie wprowadza takiego
ograniczenia. Warstwa ta b�dzie s�u�y�a do administrowania systemem,
przegl�dania podj�tych decyzji i~zarz�dzania ankietami i~u�ytkownikami.
\index{webserwisy}
\index{Apache Axis}
\paragraph{Apache Axis2}
Axis2 to kontener webserwis�w odpowiadaj�cy za odbieranie zdalnych wywo�a�
procedur i~przekazywanie ich do mog�cych je wykona� obiekt�w.
Zwalnia programist� z konieczno�ci samodzielnego przetwarzania XML w
��daniach, umo�liwia te� automatyczn� serializacj� i~deserializacj� obiekt�w
Javowych. 
\index{XSLT}
\paragraph{Transformata XSLT}
Elementem wspomagaj�cym w systemie jest transformata XSLT, wykonywana na
odpowiedziach, przesy�anych do urz�dzenia mobilnego. Jej celem jest utworzenie
widoku dla urz�dzenia mobilnego w spos�b niezale�ny od generowanej odpowiedzi.
W wyniku wywo�ania procedury generuj�cej ekran tworzony jest XML zawieraj�cy
podstawowe informacje, a nast�pnie przetwarzany jest on w opisany spos�b,
dzi�ki czemu odpowied� mo�na dowolnie modyfikowa� bez zmiany kodu aplikacji.

  

\subsection{Aplikacja mobilna}

Cech� wyr�niaj�c� podej�cie prezentowane w niniejszej pracy od typowych
rozwi�za� problemu integracji system�w enterprise jest spos�b generowania
interfejsu u�ytkownika. W typowym podej�ciu po stronie aplikacji mobilnej mamy
do czynienia z raz zdefiniowanym przez programist� uk�adem oraz wygl�dem
interfejsu graficznego. Wi��e si� to bezpo�rednio ze sposobem pisania aplikacji,
w kt�rym programista ma do dyspozycji gotowe komponenty UI, kt�re musz� zosta�
odpowiednio oprogramowane i wkompilowane w ko�cow� wersj� aplikacji mobilnej.
To powoduje, �e jakiekolwiek zmiany w interfejsie s� mo�liwe tylko i wy��cznie
poprzez przebudow� aplikacji. A co za tym idzie tak�e i ponowne rozprowadzenie
tak zmienionego programu. Takie podej�cie generuje dodatkowe koszty, wynikaj�ce
z potrzeby zapewnienia dostarczenia aktualnej wersji aplikacji do wszystkich
klient�w, posiadaj�cych jej star� wersj�. Cz�ciowo problem ten rozwi�zuje
podej�cie, anga�uj�ce cienkiego klienta w postaci mobilnej przegl�darki
internetowej. Zapewnia ono centralizacj� aplikacji, przez co wdro�enie nowych
wersji staje si� niemal automatyczne. Niestety, rozwi�zanie tego typu jest
cz�sto niewystarczaj�co elastyczne oraz posiada s�abo rozbudowany interfejs
u�ytkownika. 
W niniejszej pracy wykorzystane zosta�o podej�cie po�rednie, b�d�ce czym�
pomi�dzy interfejsem generowanym przez przegl�darki internetowe, a
pe�nowarto�ciowym interfejsem aplikacji mobilnych. 
\subsubsection{Dynamicznie generowany interfejs u�ytkownika}
W odr�nieniu od typowej aplikacji dost�pnej na urz�dzeniach mobilnych, nasza
aplikacja nie posiada zdefiniowanego z g�ry interfejsu u�ytkownika. Za
wyj�tkiem g��wnego ekranu, wszystkie pozosta�e s� generowane na podstawie
danych, przesy�anych z serwera. System zachowuje si� analogicznie do
przegl�darki internetowej - renderuje znaczniki interfejsu u�ytkownika. W
odr�nieniu od rozwi�za� opartych na przegl�darce, mamy mo�liwo�� zmian w
kodzie �r�d�owym aplikacji renderuj�cej, dzi�ki czemu, w zale�no�ci od
potrzeby, mo�emy dostosowywa� zachowanie programu do wymaga� u�ytkownika. Nasza
aplikacja nie jest ograniczona przez niemodyfikowaln� przegl�dark� internetow�.
\index{Kuix}
\subsubsection{Kuix}
Opisany w poprzednim rozdziale dynamicznie generowany interfejs jest mo�liwy do
osi�gni�cia w �rodowisku J2ME przy u�yciu pewnego, danego z g�ry, uniwersalnego
sposobu opisu wygl�du poszczeg�lnych element�w aplikacji. Ze wzgl�du na
popularno�� XML oraz wsparcie dla tego j�zyka, zar�wno po stronie urz�dze�, jak
i serwer�w, mo�na przyj��, �e w�a�nie w nim opisany zostanie interfejs
u�ytkownika. Po stronie serwera, na podstawie informacji z bazy danych, budowany
b�dzie dokument XML, ��cz�cy dane z ich sposobem wizualizacji. Po stronie
urz�dzenia mobilnego, dokument ten zostanie przetworzony przez parser XML, a
nast�pnie jego poszczeg�lne elementy zostan� zmapowane na odpowiadaj�ce im
elementy interfejsu J2ME (odpowiednio rozszerzone o dodatkowe komponenty,
niedost�pne w standardowej wersji tego �rodowiska). Opisana powy�ej procedura
zosta�a zaimplementowana w szkielecie programistycznym kuix.
\subsubsection{Cechy szkieletu Kuix}
Model programistyczny, oferowany przez szkielet Kuix, znacznie r�ni si� od
innych rozwi�za� spotykanych na platformach mobilnych. Posiada on wiele cech
charakterystycznych dla lekkiego modelu tworzenia oprogramowania mobilnego -
opartego o mobiln� przegl�dark� internetow�. Najlepsz� ilustracj� zasady
dzia�ania tego szkieletu, jest przyk�adowa aplikacja, wi�c w dalszej cz�ci
pracy zostanie przedstawiony kr�tki, przyk�adowy program, obrazuj�cy podstawowe
idee.

\paragraph{HelloKuix}

Na rysunku \ref{fig:kuix_structure} widoczna jest struktura bardzo prostego programu,
posiadaj�cego wszystkie cechy dostarczane przez szkielet Kuix.

\begin{figure}[htb]
    \begin{center}
    \includegraphics[angle=0,scale=0.7]{img/kuix_structure.png}
    \end{center}
    \caption{Struktura prostego projektu wykorzystuj�cego szkielet Kuix}
    \label{fig:kuix_structure}
\end{figure}


Poza standardowymi plikami *.java zawieraj�cymi kod aplikacji nale�y tu zwr�ci�
uwag� na dodatkowe pliki zasob�w :
\begin{itemize}
  \item helloworld.css
  \item helloworld.xml
\end{itemize}

Plik helloworld.xml to dokument XML we wspomnianym wcze�niej formacie,
reprezentuj�cym dane wraz ze sposobem ich wizualizacji. 

\begin{verbatim}
<screen title="helloworld">
    <container 
		style="layout:inlinelayout(false,fill); align: center">
        	<text text="Hello World!" />
        	<picture src="logo_community.png" />
    </container>
</screen>
\end{verbatim}

Tag screen to pojedynczy ekran na urz�dzeniu mobilnym, mog�cy posiada� jeden,
b�d� wi�cej, tak zwanych widget�w - element�w interfejsu u�ytkownika. Powy�szy
przyk�ad zawiera kontener (container), czyli pojemnik widget�w, pozwalaj�cy na
grupowanie ich w odr�bne uk�ady. W pojemniku znajduje si� tekst (tag text) oraz
obazek (tag picture). Informacje o zawarto�ci oraz cechach
poszczeg�lnych element�w przekazywane s� za pomoc� atrybut�w.

Powy�szy plik xml jest wczytywany na pocz�tku dzia�ania programu. S�u�y do tego
kod widoczny poni�ej.

\begin{verbatim}
// Load the content from the XML 
// file with Kuix.loadScreen static method
Screen screen = Kuix.loadScreen("helloworld.xml", null);

// Set the application current screen  
screen.setCurrent();
\end{verbatim}

Jak wida� z za��czonego przyk�adu, budowanie element�w interfejsu oraz ich
wczytywanie jest bardzo proste i w pewien spos�b przypomina tworzenie
interfejsu u�ytkownika dla stron wy�wietlanych w przegl�darkach inernetowych.
\index{CSS}
\paragraph{CSS, a Kuix}
Charakterystycznym elementem zapo�yczonym przez Kuix z technologii
internetowych jest spos�b nadawania po��danego wygl�du elementom aplikacji.
S�u�� do tego kaskadowe arkusze styl�w (Cascading Style Sheets). W przyk�adzie
z poprzedniego paragrafu plik helloworld.css wygl�da nast�puj�co :

\begin{verbatim}
text {
    align: center;
    font-style: normal;
    color: #f19300;
}
screenTopbar text {
    color: white;
    padding: 1 2 1 2;
}
screenTopbar {
    font-style: bold;
    bg-color: #cccccc;
    border: 0 0 1 0;
    border-color: #f19300;
}
desktop {
    bg-color: #444447;
}
\end{verbatim}

Powy�szy kod CSS praktycznie niczym si� nie r�ni od tego, spotykanego w
projektach pisanych pod standardowe przegl�darki internetowe.
Do ka�dego elementu interfejsu, takiego jak ekran, czy pulpit (screen, desktop)
mamy przypisane odpowiednie selektory. Na przyk�ad, by nada� styl paskowi
tytu�owemu ekranu, u�ywamy selektora screenTopbar.

By za�adowa� do systemu plik CSS, znajduj�cy si� w zasobach projektu, nale�y
u�y� statycznej metody loadCSS klasy Kuix:

\begin{verbatim}
// Load the stylesheet from the CSS-like file with 
// Kuix.loadCss static method
//  note: a stylesheet is not associated with 
//		  a screen but with the midlet
//  note 2: by default '/css/' folder is use 
//			to find the 'helloworld.css' file
Kuix.loadCss("helloworld.css");
\end{verbatim}

Ko�cowy efekt po��czenia pliku XML ze stylami CSS widoczny jest na rysunku
\ref{fig:kuix_helloworld}.

\begin{figure}[htb]
    \begin{center}
    \includegraphics[angle=0,scale=0.7]{img/kuix_helloworld.png}
    \end{center}
    \caption{Wygl�d omawianej aplikacji}
    \label{fig:kuix_helloworld}
\end{figure}


W analogiczny spos�b mo�na r�wnie �atwo doda� podr�czne menu do aplikacji :

\begin{verbatim}
    <screenFirstMenu>Exit</screenFirstMenu>
    <screenSecondMenu>
        more...
        <menuPopup>
            <menuItem>
                About
            </menuItem>
            <menuItem>
                Exit
            </menuItem>
        </menuPopup>
    </screenSecondMenu> 
\end{verbatim}

Efekt ko�cowy wraz z menu dost�pnym pod prawym przyciskiem (tak zwane Second
Menu) widzimy na rysunku \ref{fig:kuix_helloworld2}

\begin{figure}[htb]
    \begin{center}
    \includegraphics[angle=0,scale=0.7]{img/kuix_helloworld2.png}
    \end{center}
    \caption{Wygl�d omawianej aplikacji}
    \label{fig:kuix_helloworld2}
\end{figure}


\paragraph{Problemy}

W chwili pisania niniejszej pracy �rodowisko Kuix by�o nowo�ci� na rynku.
Najbardziej aktualna wersja 1.01 zawiera nadal du�o wad i b��d�w. Z wykonanych
przez nas test�w wynika, �e najwi�ksze problemy zwi�zane s� z pewnymi
przek�amaniami graficznymi, kt�re mo�na spotka� na urz�dzeniach mobilnych firmy
Nokia. B��dy te nie utrudnia�y pracy, a jedynie pozostawia�y wra�enie
og�lnego niedopracowania obecnej wersji szkieletu. Nale�y spodziewa� si�, �e
zostan� usuni�te w nast�pnych wersjach. 
\newline
Kolejnym powa�nym problem zwi�zanym ze szkieletem jest uboga dokumentacja
techniczna. Dobrze i szczeg�owo wykonana jest jedynie dokumentacja w
formie javadoc oraz kursy wprowadzaj�ce do tematyki. Niestety, na ich podstawie
mo�na opanowa� jedynie elementarne zasady pos�ugiwania si� �rodowiskiem. Jednak,
dzi�ki dost�pno�ci kodu �r�d�owego (Kuix jest w pe�ni otwarty), mo�liwe jest
zapoznanie si� z nieudokumentowanymi funkcjami oraz ewentualne
wprowadzenie w�asnych poprawek.
\newline
Bardzo uci��liwym problemem jest rozmiar skompilowanych bibliotek �rodowiska
Kuix. Na chwil� obecn� jest to oko�o 250 kilobajt�w. Niestety wiele urz�dze�,
w kt�re zosta�a wbudowana wirtualna maszyna Java, ogranicza dopuszczalny
rozmiar plik�w jar. Najcz�ciej ograniczenie to oscyluje w okolicach 100-150 kilobajt�w,
przez co niemo�liwe jest wykorzystanie Kuix na tych platformach. Problem ten dotyczy g��wnie rozwi�za� konsumenckich, gdy� w rozwi�zaniach mobilnych dla biznesu ograniczenia s� du�o wy�sze lub mo�liwe jest dzielenie programu na biblioteki(takie rozwi�zanie mo�na znale�� na
platformie Blackberry).

\section{Przyk�adowa implementacja - Mobilny system informacyjny}

\subsection{Koncepcja}

\subsection{Interfejs mobilny}

\subsection{Interfejs administracyjny}

 

\chapter{Zako�czenie}
Rozw�j urz�dze� mobilnych oraz coraz szybszych ��czy
komunikacyjnych przyczynia si� do powstawania mo�liwo�ci uniezale�nienia si� od
jednego miejsca pracy. Stale powi�ksza si� grono os�b, kt�re doceniaj�
mo�liwo�ci oferowane przez organizery, PDA i smartfony. Pr�ba przeniesienia
funkcjonalno�ci, wcze�niej zarezerwowanych dla klasycznych komputer�w na
urz�dzenia mobilne rodzi nowe wyzwania, kt�re starali�my si� opisa� w naszej pracy. \\
\indent
Przedstawienie wszystkich zagadnie� zwi�zanych z tworzeniem i integracj�
mobilnych aplikacji nie by�oby mo�liwe. W zwi�zku z tym, podj�li�my pr�b�
przedstawienia najbardziej typowych problem�w, a tak�e ich uznanych rozwi�za� -
sposob�w tworzenia mobilnych aplikacji. 
\\
Wyznaczony w ten spos�b cel pracy zosta� zrealizowany w dw�ch pocz�tkowych
rozdzia�ach. Rozdzia� pierwszy stanowi wprowadzenie do zagadnienia integracji mobilnej.
Stawia pytania i prezentuje problemy, kt�re uznali�my za najbardziej kluczowe
dla tej dziedziny. Rozdzia� drugi przedstawia podej�cia do tworzenia
mobilnych aplikacji integruj�cych. Stanowi� one pewn� metodyk�, kt�ra powsta�a
w wyniku poszukiwania rozwi�za� problem�w przedstawionych w pierwszym
rozdziale. Jej stosowanie pozwala znacznie skr�ci� czas budowania nowych
aplikacji oraz rozwi�za� cz�� klasycznych problem�w integracji.\\
\indent
R�norodno�� mobilnych platform rodzi konieczno�� skoncentrowania si� na
jednej z nich, aby by� w stanie tworzy� na niej aplikacje. W naszej
pracy zdecydowali�my si� na wykorzystanie Javy w celu stworzenia szablonu
integracyjnego, przedstawionego w trzecim rozdziale. Pragn�li�my tak�e
zaprezentowa� nieco odmienne podej�cie do tworzenia aplikacji mobilnych ni�
to, obecnie powszechnie stosowane. Przedstawiona idea mo�e zosta� wykorzystana
na innych platformach, co stworzy�oby mo�liwo�ci tworzenia bardziej
interaktywnych, dostosowanych do urz�dze� mobilnych, aplikacji.\\
\indent
W niniejszej pracy pokazali�my, �e integracja mobilna stanowi nowe wyzwanie,
kt�re jednak bardzo du�o wsp�lnego ma ze �wiatem integracji klasycznej.
Wi�kszo�� generowanych przez integracj� mobiln� problem�w mo�na r�wnie�
odnale�� w �wiecie integracji klasycznej, a to oznacza, �e tak�e tam mo�na
szuka� ich rozwi�za�. Nie nale�y jednak zapomina� o tym, �e �rodowisko mobilne
jest ograniczone i na chwil� obecn� nadal oferuje tylko u�amek wydajno�ci
stacjonarnych �rodowisk komputerowych. Nie mo�na r�wnie� za�o�y�, �e
kiedykolwiek mo�liwo�ci tych �rodowisk si� zr�wnaj� - wraz z rozwojem urz�dze�
mobilnych nast�puje rozw�j pozosta�ych �rodowisk. W zwi�zku z tym, warto
po�wi�ci� czas i przygotowa� nowe metodologie tworzenia aplikacji, kt�re nie b�d� tylko i
wy��cznie kalk� rozwi�za� spotykanych w �rodowiskach Enterprise.
  
 
\begin{thebibliography}{99}
\bibitem{Entj2me} Michael Juntao Yuan: \emph{Enterprise J2ME Developing Mobile
Applcations}, Prentice Hall Pennsylvania 2004
\end{thebibliography} 
 
\printindex

\appendix
 
% tutaj za��czniki

%\chapter*{Bibliografia}
%\nocite{*}
%\bibliographystyle{plplain}
%\bibliographystylebk{plplain}
%\bibliographystylest{plplain}
%\bibliographystyledoc{plplain}
% \bibliographystyleweb{plplain}
%\bibliographybk{BIB/books}
%\bibliographyst{BIB/books}
%\bibliographydoc{BIB/books}
% \bibliographyweb{BIB/books}

% \bibliography{bib/verificard,bib/jml,bib/daikon}
%\bibliography{bib/daikon,bib/statistics,bib/other}

\end{document}

% ex: set tabstop=4 shiftwidth=4 softtabstop=4 noexpandtab fileformat=unix filetype=tex encoding=utf-8 fileencodings= fenc= spelllang=pl,en spell:

%%This is a very basic article template.
%%There is just one section and two subsections.
%%\documentclass[12pt, a4paper]{report}
%%\linespread{1.5}
%%\pagestyle{headings}
%%\usepackage{fancyhdr}   
%%\usepackage{graphicx}     
%%\pagestyle{fancy}           
%zmianaliterw~ywejpaginienamae 
%%\renewcommand{\chaptermark}[1]{\markboth{#1}{}}
%%\renewcommand{\sectionmark}[1]{\markright{\thesection\ #1}}
%%\fancyhf{}%usubie�ceustawieniapagin 
%%\fancyhead[LE,RO]{\small\bfseries\thepage}
%%\fancyhead[LO]{\small\bfseries\rightmark}    
%%\fancyhead[RE]{\small\bfseries\leftmark}  
%%\renewcommand{\headrulewidth}{0.5pt}  
%%\addtolength{\headheight}{0.5pt}%pionowyodstpnakresk
%%\fancypagestyle{plain}{%  
%%\fancyhead{}%usup.g�rnenastronachpozbawionych 
%numeracji(plain) 
%%\renewcommand{\headrulewidth}{0pt}%poziomakreska
%%}
%%\usepackage{graphicx}  
%%\usepackage{polski}
%%\usepackage[cp1250]{inputenc} 
%%\end{document}
 